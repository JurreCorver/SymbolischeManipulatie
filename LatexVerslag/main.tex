\documentclass[a4paper]{article}
%\usepackage[dutch]{babel}
\usepackage[utf8x]{inputenc}
\usepackage{amsmath}
\usepackage{graphicx}
\usepackage{hyperref}
\usepackage[colorinlistoftodos]{todonotes}

\title{Symbolische manipulatie}
\author{Rik Voorhaar(3888169) - Jan-Willem van Ittersum(3992942) - Jurre Corver(3905985)}

\begin{document}
\maketitle
\clearpage

%\begin{abstract}
%Your abstract.
%\end{abstract}

\section{Introductie}
In deze opdracht hebben we een eigen computeralgebrasysteem (CAS) ontwikkeld om symbolisch te kunnen rekenen zoals dat bijvoorbeeld in Mathematica gebeurt. Wiskundige formules worden hiervoor opgeslagen in een zogenaamde expressie-boom. Behalve dat deze representatie kan worden gebruikt om berekeningen te doen, is deze geschikt voor symbolische manipulaties, zoals optellen, vermenigvuldigen, maar ook differenti\"eren en oplossen van sommige polynoom~vergelijkingen. De code behordende bij dit project kan gevonden worden op \url{https://github.com/JurreCorver/SymbolischeManipulatie}.


\section{Theorie}
%Beschrijving van de theorie


\section{Algoritmen}
%Beschrijving van de gebruikte algoritmen, denk hierbij ook aan bijvoorbeeld de complexiteit en de gebruikte datastructuren



\section{Documentatie}
%Documentatie over het gebruik van de code

\section{Resultaten}
%Eventuele resultate die behaald zijn met de code

\section{Taakverdeling}
%De taakverdeling binnen je groepje.




%%%% Make bibliography
%%



\end{document}